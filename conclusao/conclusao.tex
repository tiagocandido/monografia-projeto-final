%%%%%%%%%%%%%%%%%%%%%%%%%%%%%%%%%%%%%%%%%%%%%%%%%%%%%%%%
%                      Conclus�o                       %
%%%%%%%%%%%%%%%%%%%%%%%%%%%%%%%%%%%%%%%%%%%%%%%%%%%%%%%%

\chapter{Conclus�o}
\thispagestyle{empty} % retira numeracao da pagina, conforme as normas de apresentacao.

A conclus�o � a finaliza��o do trabalho textual. Com base nas argumenta��es dos itens do desenvolvimento, ser� realizada a conclus�o das id�ias apresentadas na monografia.

Nesta etapa deve ser usada uma linguagem mais direta, visando � persuas�o do leitor. Al�m disso, � importante evitar per�odos muito longos e fazer uso de conectivos para juntar as id�ias, a fim de tornar o texto o mais l�gico poss�vel. Uma conclus�o fraca arrasar� sua monografia, pois um trabalho sem um ponto final � um trabalho inacabado. Portanto, muita aten��o com a conclus�o, pois ela pode decidir se seu trabalho foi bem-sucedido ou n�o.

Enfim, a conclus�o � a converg�ncia das id�ias tratadas em toda sua monografia, visto que h� um encaminhamento das mesmas, 
atrav�s de pensamentos l�gicos, para uma defini��o.
