%%%%%%%%%%%%%%%%%%%%%%%%%%%%%%%%%%%%%%%%%%%%%%%%%%%%%%%%
%                      Conclusão                       %
%%%%%%%%%%%%%%%%%%%%%%%%%%%%%%%%%%%%%%%%%%%%%%%%%%%%%%%%

\chapter{Conclusão}
\thispagestyle{empty} % retira numeracao da pagina, conforme as normas de apresentacao.

O trabalho final tem como resultado uma aplicação híbrida móvel e uma API integrada com o Conexão UFF. Tais aplicações permitem que os alunos da universidade se conectem ao sistema do Conexão UFF e recebam atualizações de disciplinas, tópicos, eventos e arquivos através da aplicação de uma forma simples e prática.

Considerando os sistemas atuais, é possível identificar melhorias e novas funcionalidades que podem ser implementadas em trabalhos futuros. A principal delas sendo a integração com outras plataformas utilizadas por alunos e professores dentro da UFF. Já que o maior objetivo deste trabalho é facilitar a integração de diversas plataformas diferentes.

Outro exemplo de uma nova funcionalidade seria a implementação de um sistema de mensagens entre alunos através do IntegraUFF. Dessa forma os usuários não precisariam se conectar diretamente através das interfaces de cada plataforma integrada para se comunicar com os outros usuários, e as mensagens ficariam centralizadas em um único local.	

Atualmente o IntegraUFF só permite a sincronização e leitura das informações obtidas das plataformas integradas. Seria desejável que também fosse possível a escrita de informações em tais plataformas através da aplicação. Para a implementação desta funcionalidade é importante considerar que é necessário verificar se as interfaces das plataformas permitem	tal tipo de integração. 

Em relação as melhorias de funcionalidades já existentes, o IntegraUFF hoje já possui uma listagem de eventos na qual é possível marcá-los na agenda padrão do celular. Seria interessante extender esta integração para que o calendário fosse exibido de dentro do aplicativo e/ou os eventos pudessem ser marcados na agenda automaticamente. E para os arquivos, uma integração com o serviço do \textit{Dropbox} \cite{website:dropbox}, permitindo que os alunos enviem os arquivos para a nuvem, e se possível que isso fosse feito de forma automática. 
			
O IntegraUFF é um projeto de código aberto, tanto a sua parte mobile quanto o servidor. Os códigos podem ser obtidos através dos links https://github.com/tiagocandido/client-side-integra-uff, para a aplicação móvel, e https://github.com/tiagocandido/server-side-integra-uff, para o servidor. 		

