%%%%%%%%%%%%%%%%%%%%%%%%%%%%%%%%%%%%%%%%%%%%%%%%%%%%%%%%
%                        Texto                         %
%%%%%%%%%%%%%%%%%%%%%%%%%%%%%%%%%%%%%%%%%%%%%%%%%%%%%%%%

\chapter{Introdu��o}
\thispagestyle{empty} % retira numeracao da pagina, conforme as normas de apresentacao.

\section{Problema}

A cria��o do sistema foi motivada pelo problema da descentraliza��o das informa��es e conte�dos das disciplinas da universidade. At� o presente momento tais informa��es est�o presentes em diversas plataformas de ensino diferentes, dificultando o acesso dos alunos e professores que as utilizam.

\section{Prop�sito}

Este documento descreve o projeto de aplicativo proposto pelos autores para a disciplina de Projeto de Aplica��o I, especificando os requisitos da sua vers�o inicial.

\section{P�blico-Alvo e Sugest�es para Leitura}

Este documento se destina aos pr�prios autores e ao orientador do projeto como material de apoio ao desenvolvimento e gerenciamento das atividades. Tamb�m destina-se a outros desenvolvedores que visem estender o escopo do projeto.

\section{Escopo do Produto}

O produto se trata de um aplicativo m�vel multi-plataforma, que tem como prop�sito reunir, em uma �nica ferramenta, m�ltiplos ambientes de ensino, apresentando de maneira transparente ao usu�rio os servi�os oferecidos pelos diferentes ambientes de ensino, como eventos, t�picos, arquivos, mensagens, todos em uma �nica interface.
O uso do aplicativo tem como objetivo aproximar alunos e professores das diversas disciplinas ministradas na universidade, intensificar a intera��o das turmas, centralizar as informa��es e materiais da disciplina e prover uma comunica��o r�pida e eficiente. Al�m disso, a ferramenta  prop�e a ludifica��o das atividades no grupo, atrav�s de desafios, marcos, medalhas e outros items que possam ser utilizados pelo professor da disciplina para gerar indicadores que permitam avaliar a participa��o dos alunos no grupo.
