\chapter{Sistemas de Gest�o de Aprendizado}
\thispagestyle{empty} % retira numeracao da pagina, conforme as normas de apresentacao.

O desenvolvimento � a parte na qual ser�o feitas as argumenta��es de sua tese e ser�o confrontados seus dados para uma futura conclus�o das id�ias. Corresponde � etapa mais trabalhosa de sua monografia e � mais importante tamb�m.

Os par�grafos podem ser curtos ou longos, dependendo das id�ias apresentadas neles. Entretanto, par�grafos longos exigem certa aten��o extra, pois podem ficar cansativos e prolixos. Por conta disso, � aconselh�vel que o desenvolvimento seja escrito inicialmente como rascunho. Fa�a a revis�o deste algumas vezes, para aprimor�-lo, uma vez que, sendo esta parte longa, dificilmente ser� escrita perfeitamente em um primeiro momento.
	
Portanto, o desenvolvimento � o centro das argumenta��es de sua tese. Nele ser�o expostas id�ias que contribuir�o para a persuas�o do leitor.


