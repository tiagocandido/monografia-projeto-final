%%%%%%%%%%%%%%%%%%%%%%%%%%%%%%%%%%%%%%%%%%%%%%%%%%%%
%            Resumo na l�ngua vern�cula            %
%%%%%%%%%%%%%%%%%%%%%%%%%%%%%%%%%%%%%%%%%%%%%%%%%%%%

\chapter*{Resumo}
\addcontentsline{toc}{chapter}{Resumo}

\thispagestyle{myheadings}

As disciplinas dos cursos da Universidade Federal Fluminense possuem informa��es alocadas em diversos Sistemas de Gest�o de Aprendizado. Desta forma a tarefa de gerenciar os conte�dos disponibilizados pelos professores pode se tornar bastante custosa. 
Com o objetivo de diminuir o trabalho manual realizado pelos alunos, propomos a cria��o de uma aplica��o que possibilite agregar as informa��es das v�rias disciplinas em uma �nica interface de usu�rio, independente de onde ela foi disponibilizada pelo professor.
Para alcan�ar este objetivo, desenvolvemos um aplicativo mo?vel multi�plataforma, que tem como propo?sito reunir, em uma u?nica ferramenta, mu?ltiplos ambientes de ensino, apresentando de maneira transparente ao usua?rio os servic?os oferecidos pelos diferentes ambientes de ensino, como eventos, to?picos, arquivos, mensagens, todos em uma u?nica interface. 
	O uso do aplicativo tem como objetivo aproximar alunos e professores das diversas disciplinas ministradas na universidade, intensificar a interac?a?o das turmas, centralizar as informac?o?es e materiais da disciplina e prover uma comunicac?a?o ra?pida e eficiente. \\

Palavras-chave: Sistema de Gest�o de Aprendizado, Aplica��o Mobile, Phonegap, Ruby on Rails.

