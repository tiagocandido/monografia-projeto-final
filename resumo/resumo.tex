%%%%%%%%%%%%%%%%%%%%%%%%%%%%%%%%%%%%%%%%%%%%%%%%%%%%
%            Resumo na língua vernácula            %
%%%%%%%%%%%%%%%%%%%%%%%%%%%%%%%%%%%%%%%%%%%%%%%%%%%%

\chapter*{Resumo}
\addcontentsline{toc}{chapter}{Resumo}

\thispagestyle{myheadings}

As disciplinas dos cursos da Universidade Federal Fluminense possuem informações alocadas em diversos Sistemas de Gestão de Aprendizado. Desta forma a tarefa de gerenciar os conteúdos disponibilizados pelos professores pode se tornar bastante custosa. 

Com o objetivo de diminuir o trabalho manual realizado pelos alunos, propomos a criação de uma aplicação que possibilite agregar as informações das várias disciplinas em uma única interface de usuário, independente de onde ela foi disponibilizada pelo professor.

Para alcançar este objetivo, desenvolvemos um aplicativo móvel multi-plataforma, que tem como propósito reunir, em uma única ferramenta, múltiplos ambientes de ensino, apresentando de maneira transparente ao usuário os serviços oferecidos pelos diferentes ambientes de ensino, como eventos, tópicos, arquivos, mensagens, todos em uma única interface. 

O uso do aplicativo tem como objetivo aproximar alunos e professores das diversas disciplinas ministradas na universidade, intensificar a interação das turmas, centralizar as informações e materiais da disciplina e prover uma comunicacção rápida e eficiente.

Palavras-chave: Sistema de Gestão de Aprendizado, Aplicação Mobile, Phonegap, Ionic, Padrões de Projeto, Ruby on Rails.

